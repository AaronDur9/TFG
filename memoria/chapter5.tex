\cleardoublepage

\chapter{Instalación y puesta en marcha}
\label{makereference5}

\section{Nodo}
\label{makereference5.1}
\subsection{Requisitos}
\label{makereference5.1.2}
	\begin{itemize}
		\item \textbf{Raspberry Pi 2 model B} con una distribución Linux (\href{https://www.raspberrypi.org/downloads/raspbian/}{Raspbian})
		\item Conexión a Internet
		\item \href{https://www.adafruit.com/product/385}{Sensor DHT22} para medir humedad y temperatura
			\begin{itemize}
				\item Pin 1 a vcc (3.3V)
				\item Pin 2 a vcc (3.3V) con 10K 0hms de resistencia entre él y GPIO04
				\item Pin 3 desconectado
				\item Pin 4 a GND
			\end{itemize}
		\item \href{https://www.adafruit.com/product/856}{MCP3008}
		\item \href{https://www.apogeeinstruments.co.uk/content/SP-212-215-manual.pdf}{Piranómetro SP-212}
		\item SPI activo en la Raspberry con ``raspi-config''
	\end{itemize}

Las conexiones vienen explicadas en:
\lstset{language=bash}
\begin{lstlisting}[frame=single]
$ node/node_setup.txt
\end{lstlisting}

\subsubsection{Software}
	En la Raspberry Pi necesitamos
	\begin{itemize}
		\item \textbf{paho-mqtt} (\cite{ARP:Paho:2017})
	\end{itemize}

\subsection{Ficheros}
\label{makereference5.1.3}
	\begin{itemize}
		\item Introduce el directorio \textbf{node} en tu Rapsberry Pi
		\item Cambia las variables \textbf{brokerIp, brokerPort, topic, ubication} en solar-node.py
	\end{itemize}

\subsection{Dependencias}
\label{makereference5.1.4}
	\begin{itemize}
		\item Actualizar
\lstset{language=bash}
\begin{lstlisting}[frame=single]
$ sudo apt-get update
$ sudo apt-get upgrade
\end{lstlisting}
		\item Python 2.7 (\cite{ARP:Python:2017})
\begin{lstlisting}[frame=single]
$ sudo apt-get install python2.7 build-essential 
python-pip python-dev
\end{lstlisting}
		\item Paho (\cite{ARP:Paho:2017})
\begin{lstlisting}[frame=single]
$ pip install paho-mqtt
\end{lstlisting}
		\item WiringPi (\cite{ARP:Wiring:2017})
\begin{lstlisting}[frame=single]
$ pip install wiringpi2
\end{lstlisting}
		\item Adafruit\_DHT (\cite{ARP:Adafruit:2017})
\begin{lstlisting}[frame=single]
$ git clone 
https://github.com/adafruit/Adafruit_Python_DHT.git
$ cd Adafruit_Python_DHT
$ sudo python setup.py install
\end{lstlisting}
		\item Adafruit\_GPIO
\begin{lstlisting}[frame=single]
$ git clone 
https://github.com/adafruit/Adafruit_Python_GPIO.git
$ cd Adafruit_Python_GPIO
$ sudo python setup.py install
\end{lstlisting}
		\item Adafruit\_MCP3008
\begin{lstlisting}[frame=single]
$ git clone 
https://github.com/adafruit/Adafruit_Python_MCP3008.git
$ cd Adafruit_Python_MCP3008
$ sudo python setup.py install
\end{lstlisting}
	\end{itemize}

\section{Servidor de datos}
\label{makereference5.2}
\subsection{Requisitos previos}
\begin{itemize}
\item Máquina con distribucion Linux instalada. Recomendable Debian, Fedora, OpenSUSE o Ubuntu.
\item conexión a internet.
\item IP fija y opcionalmente, nombre de dominio.
\end{itemize}

La instalación de un broker MQTT es muy sencilla (una de sus principales ventajas). Para ello basta con instalar el demonio mosquitto a través del siguiente comando:

Ejemplo de instalación en Ubuntu:
\lstset{language=bash}
\begin{lstlisting}[frame=single]
$ sudo apt-get install mosquitto
\end{lstlisting}

Por defecto, Ubuntu arranca el servicio después de instalarlo. Si la distribución Lunix sobre la cual se instala no realiza esta acción por defecto, se deberá configurar debidamente para arrancarlo.

Hablar aquí de como securizar ubuntu y cambiar la configuración por defecto.
Aquí más info: https://www.digitalocean.com/community/tutorials/how-to-install-and-secure-the-mosquitto-mqtt-messaging-broker-on-ubuntu-16-04