\cleardoublepage

\chapter{Instalación y puesta en marcha}
\label{makereference5}

\section{Nodo}
\label{makereference5.1}
\subsection{Requisitos}
\label{makereference5.1.2}
	\begin{itemize}
		\item \textbf{Raspberry Pi 2 model B} con una distribución Linux (\href{https://www.raspberrypi.org/downloads/raspbian/}{Raspbian})
		\item Conexión a Internet
		\item \href{https://www.adafruit.com/product/385}{Sensor DHT22} para medir humedad y temperatura
			\begin{itemize}
				\item Pin 1 a vcc (3.3V)
				\item Pin 2 a vcc (3.3V) con 10K 0hms de resistencia entre él y GPIO04
				\item Pin 3 desconectado
				\item Pin 4 a GND
			\end{itemize}
		\item \href{https://www.adafruit.com/product/856}{MCP3008}
		\item \href{https://www.apogeeinstruments.co.uk/content/SP-212-215-manual.pdf}{Piranómetro SP-212}
		\item SPI activo en la Raspberry con ``raspi-config''
	\end{itemize}

Las conexiones vienen explicadas en:
\lstset{language=bash}
\begin{lstlisting}[frame=single]
$ node/node_setup.txt
\end{lstlisting}

\subsubsection{Software}
	En la Raspberry Pi necesitamos
	\begin{itemize}
		\item \textbf{paho-mqtt}
	\end{itemize}


\subsection{Ficheros}
\label{makereference5.1.3}
	\begin{itemize}
		\item Introduce el directorio \textbf{node} en tu Rapsberry Pi
		\item Cambia las variables \textbf{brokerIp, brokerPort, topic, ubication} en solar-node.py
	\end{itemize}

\subsection{Dependencias}
\label{makereference5.1.4}
	\begin{itemize}
		\item Actualizar
\lstset{language=bash}
\begin{lstlisting}[frame=single]
$ sudo apt-get update
$ sudo apt-get upgrade
\end{lstlisting}
		\item Python 2.7
\begin{lstlisting}[frame=single]
$ sudo apt-get install python2.7 build-essential 
python-pip python-dev
\end{lstlisting}
		\item Paho
\begin{lstlisting}[frame=single]
$ pip install paho-mqtt
\end{lstlisting}
		\item WiringPi
\begin{lstlisting}[frame=single]
$ pip install wiringpi2
\end{lstlisting}
		\item Adafruit\_DHT
\begin{lstlisting}[frame=single]
$ git clone 
https://github.com/adafruit/Adafruit_Python_DHT.git
$ cd Adafruit_Python_DHT
$ sudo python setup.py install
\end{lstlisting}
		\item Adafruit\_GPIO
\begin{lstlisting}[frame=single]
$ git clone 
https://github.com/adafruit/Adafruit_Python_GPIO.git
$ cd Adafruit_Python_GPIO
$ sudo python setup.py install
\end{lstlisting}
		\item Adafruit\_MCP3008
\begin{lstlisting}[frame=single]
$ git clone 
https://github.com/adafruit/Adafruit_Python_MCP3008.git
$ cd Adafruit_Python_MCP3008
$ sudo python setup.py install
\end{lstlisting}
	\end{itemize}

\section{Servidor}
\label{makereference5.2}
\subsection{Prerrequisitos}
\label{makereference5.2.1}
	\begin{itemize}
		\item Cuenta en \textbf{ThinkSpeak}
		\item \textbf{Mosquitto}
	\end{itemize}
\subsection{Python version and libraries}
\label{makereference5.2.2}
	\begin{itemize}
		\item Python 2.7
		\item Pandas para manipular ficheros CSV
\begin{lstlisting}[frame=single]
$ sudo pip install pandas
\end{lstlisting}
		\item Paho MQTT para crear el servidor MQTT
\begin{lstlisting}[frame=single]
$ sudo pip install paho-mqtt
\end{lstlisting}

	\end{itemize}
\subsection{Instalación}
\label{makereference5.2.3}
\begin{itemize}
	\item Introduce el directiorio \textbf{server} en tu servidor
	\item Cambia las variables \textbf{brokerIp, brokerPort, topic and thinkspeakKey} en solar-node.py
\end{itemize}
\subsection{Uso}
\label{makereference5.2.4}
\begin{itemize}
	\item Ejecuta
\begin{lstlisting}[frame=single]
$ mosquitto
$ server/main.py
\end{lstlisting}
\end{itemize}