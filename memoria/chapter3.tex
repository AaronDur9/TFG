\cleardoublepage

\chapter{Servidor de datos}
\label{makereference5}

Here are more examples of referring to previous sections.  In
Chapter~\ref{makereference} there were several sections, including
section~\ref{makereference1.1}, section~\ref{makereference1.2},
and section~\ref{makereference1.3}.

Likewise, in Chapter~\ref{makereference2}, there are
sections~\ref{makereference2.1} and ~\ref{makereference2.2}.

\section{Introducción}
\label{makereference5.1}
Como comentamos en el capitulo anterior, el \textbf{nodo} necesita una interfaz de red para mandar los datos, en este caso estamos hablando de \textbf{MQTT}. Este sera el encargado de recibir, almacenar y organizar la información que recibe del nodo.
El \textbf{cliente MQTT} lo ejecuta el nodo y por otro lado el \textbf{servidor MQTT} esta en una máquina física que hemos alojado en la Facultad de Físicas, a la cual accedemos por SSH remotamente.

\section{MQTT}
\label{makereference5.2}
\textbf{MQTT} o lo que es lo mismo \textit{Message Queue Telemetry Transpor}t es un protocolo para la comunicación \textbf{machine-to-machine} de mensajería bastante simple y ligero, orientado a conexión por \textbf{TCP/IP}. Esta diseñado principalmente para dispositivos con poco ancho de banda y con latencia baja. 

Tiene dos puertos TCP/IP reservados, el 1883 y el 8883.

\section{ThingSpeak}
\label{makereference5.3}

\section{Bridge ThingSpeak}
\label{makereference5.4}

\section{Instalación y puesta en marcha}
\label{makereference5.5}