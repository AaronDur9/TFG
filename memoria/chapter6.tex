\cleardoublepage

\chapter{Instalación y puesta en marcha}
\label{makereference6}

\section{Nodo}
\subsection{Prerrequisitos}
\subsubsection{Hardware}
	\begin{itemize}
		\item \textbf{Raspberry Pi 2 model B} con una distribución Linux (nosotros usamos Raspbian)
		\item 1 Led
			\begin{itemize}
				\item Ánodo a 220 0hms de resistencia y la resistencia a vcc (3.3V)
				\item Cátodo a GPIO27
			\end{itemize}
		\item \textbf{Sensor DHT22} para medir humedad y temperatura
			\begin{itemize}
				\item Pin 1 a vcc (3.3V)
				\item Pin 2 a vcc (3.3V) con 10K 0hms de resistencia entre él y GPIO04
				\item Pin 3 desconectado
				\item Pin 4 a GND
			\end{itemize}
		\item \textbf{Piranómetro}
	\end{itemize}

	Las conexiones vienen explicadas en:
\lstset{language=bash}
\begin{lstlisting}[frame=single]
$ node/node_setup.txt
\end{lstlisting}

\subsubsection{Software}
	En la Raspberry Pi necesitamos
	\begin{itemize}
		\item \textbf{paho-mqtt}
	\end{itemize}
	Ejemplo de nuestro cliente MQTT
\lstset{language=python}
\begin{lstlisting}[frame=single]
import paho.mqtt.client as mqtt
import json

def listen_on_connect(client, userdata, rc):
print "Connected with result code "+ str(rc)
# Subscribing in on_connect() means that if we lose the connection
# and reconnect then subscriptions will be renewed.
client.subscribe("solar")

def listen_on_message(client, userdata, msg):
print msg.topic+" "+ str(msg.payload)

def sendToBroker(brokerIp, brokerPort, payload, topic):

client = mqtt.Client()
client.connect(brokerIp, brokerPort, 60)

result = client.publish(topic, payload=payload)
if result[0] == mqtt.MQTT_ERR_NO_CONN:
return False

client.disconnect()
return True

def listenToBroker(brokerIp, brokerPort, topic):

client = mqtt.Client()
client.on_connect = listen_on_connect
client.on_message = listen_on_message

client.connect(brokerIp, brokerPort, 60)

client.loop_forever()
\end{lstlisting}

\lstset{language=bash}
\begin{lstlisting}[frame=single]
$ pip install paho-mqtt
\end{lstlisting}

\subsection{Instalación}
	\begin{itemize}
		\item Introduce el directiorio \textbf{node} en tu Rapsberry Pi
		\item Cambia las variables \textbf{brokerIp, brokerPort, topic, ubication} en solar-node.py
	\end{itemize}

\subsection{Uso}
Ejecutar:
\begin{lstlisting}[frame=single]
$ node/solar_node.py
\end{lstlisting}

\section{Servidor}
\subsection{Prerrequisitos}
	\begin{itemize}
		\item Cuenta en \textbf{ThinkSpeak}
		\item \textbf{Mosquitto}
	\end{itemize}
\subsection{Python version and libraries}
	\begin{itemize}
		\item Python 2.7
		\item Pandas para manipular ficheros CSV
\begin{lstlisting}[frame=single]
$ sudo pip install pandas
\end{lstlisting}
		\item Paho MQTT para crear el servidor MQTT
\begin{lstlisting}[frame=single]
$ sudo pip install paho-mqtt
\end{lstlisting}

	\end{itemize}
\subsection{Instalación}
\begin{itemize}
	\item Introduce el directiorio \textbf{server} en tu servidor
	\item Cambia las variables \textbf{brokerIp, brokerPort, topic and thinkspeakKey} en solar-node.py
\end{itemize}
\subsection{Uso}
\begin{itemize}
	\item Ejecuta
\begin{lstlisting}[frame=single]
$ mosquitto
$ server/main.py
\end{lstlisting}
\end{itemize}